\documentclass{dissert}
\usepackage[T2A]{fontenc}
\usepackage[utf8]{inputenc}
\usepackage[english,russian]{babel}

\usepackage{fullpage}
\usepackage{lastpage}
\usepackage{enumerate}

\usepackage[all]{xy}
% colors
\usepackage{xcolor}
\definecolor{darkgreen}{rgb}{0.0, 0.2, 0.13}
\definecolor{darkcyan}{rgb}{0.0, 0.55, 0.55}

\usepackage{geometry}
\geometry{left=2.5cm}
\geometry{right=1.0cm}
\geometry{top=2.0cm}
\geometry{bottom=2.0cm}
\renewcommand{\baselinestretch}{1.0}

\newcommand{\paragraph}[1]{\noindent\textbf{#1}\quad}


%https://tex.stackexchange.com/questions/163451/total-number-of-citations
\usepackage{totcount}
\newtotcounter{citnum} %From the package documentation
\def\oldbibitem{} \let\oldbibitem=\bibitem
\def\bibitem{\stepcounter{citnum}\oldbibitem}

\makeatletter
\long\def\@makecaption#1#2{%
  \vskip\abovecaptionskip
  \sbox\@tempboxa{#1.~#2}%
  \ifdim \wd\@tempboxa >\hsize
    #1.~#2\par
  \else
    \global \@minipagefalse
    \hb@xt@\hsize{\hfil\box\@tempboxa\hfil}%
  \fi
  \vskip\belowcaptionskip}
\makeatother

\reversemarginpar

\renewcommand{\contentsname}{Содержание}
\renewcommand{\contentsdesc}{Стр.}
\renewcommand{\chaptername}{Глава}

%%% Библиография %%%
\makeatletter
\bibliographystyle{utf8gost71u}     % Оформляем библиографию по ГОСТ 7.1 (ГОСТ Р 7.0.11-2011, 5.6.7)
\makeatother


% Нужные мне пакеты
\usepackage{amsthm}
\usepackage{graphicx}
\usepackage{amssymb}
\usepackage{amsmath}
\usepackage{graphicx}
\usepackage{subfig}
\usepackage{caption}
\usepackage{color}
\usepackage{bm}
\usepackage{tabularx}
\usepackage{url}
\usepackage{multirow}

\newtheorem{theorem}{Теорема}
\newtheorem{lemma}[theorem]{Лемма}
\newtheorem{definition}{Определение}

\usepackage{comment}
\usepackage{rotating}

\usepackage{autonum}

\begin{document}

\begin{titlepage}
%\begin{center}
%\textsc{МОСКОВСКИЙ ФИЗИКО-ТЕХНИЧЕСКИЙ ИНСТИТУТ (ГОСУДАРСТВЕННЫЙ УНИВЕРСИТЕТ)}\\
%\end{center}
%\vspace{1.5cm}
\begin{flushright}
{На правах рукописи}
\end{flushright}
\vspace{1.5cm}
\begin{center}
{Грабовой Андрей Валериевич}
\par
\vspace{2cm}
\textsc{Априорное распределение параметров \\в задачах выбора моделей глубокого обучения}
\par
\vspace{2cm}
{05.13.17~--- Теоретические основы информатики}
\par
\vspace{2cm}
{АВТОРЕФЕРАТ\\
диссертации на соискание ученой степени\\
кандидата физико-математических наук}
\end{center}
\par
\vspace{3.5cm}
\begin{center}
{Москва~--- 2022}
\end{center}
\end{titlepage}

%\clearpage\maketitle
\setcounter{page}{2}
%\pretolerance=10000
%\thispagestyle{empty}
\noindent {Работа выполнена на Кафедре интеллектуальных систем Федерального государственного автономного образовательного учреждения высшего образования <<Московский физико-технический институт (национальный исследовательский институт)>>.

\vspace{0.1cm}

%\begin{sloppy}
%\fontdimen2\font=3pt

\vskip1ex\noindent
\begin{tabularx}{\linewidth}{@{}lX@{}}
  Научный руководитель: & \textbf{Стрижов Вадим Викторович}\\
  & доктор физико-математических наук, Федеральный исследовательский~центр <<Информатика и управление>> Российской академии наук, отдел интеллектуальных систем, ведущий научный сотрудник.
  \\[2pt]
  Официальные оппоненты: & \textbf{???}\\
  & ???\\[2pt]
  & \textbf{???}\\
  & ???
\end{tabularx}
\vskip2ex\noindent

\vspace{0.2cm}
\noindent Защита состоится~???~??? 2022 года~в~??:00 на~заседании диссертационного совета Д 002.073.05 при Федеральном исследовательском центре <<Информатика и управление>> Российской академии наук (ФИЦ~ИУ~РАН) по адресу: 119333, г.\,Москва, ул.\,Вавилова, д.\,40.

\vspace{0.2cm}
\noindent С диссертацией можно ознакомиться в библиотеке Федерального государственного учреждения Федеральный исследовательский центр <<Информатика и управление>> Российской академии наук и на сайте http://www.frccsc.ru/

\vspace{0.2cm}
\noindent Автореферат разослан  \quad \quad ??? 2022 года.

\vspace{0.3cm}
\noindent И. о. ученого секретаря\\
диссертационного совета Д 002.073.05\\
д.т.н.
\hspace{12cm} И. А. Матвеев
}

\clearpage


%%%%%%%%%%%%%%%%%%%%%%%%%%%%%%%%%%%%%%%%%%%%%%%%%%%%%%%%%%%%%%%%%%%%%%%%

\pretolerance=-1

%\setcounter{page}{0}

\section*{Общая характеристика работы}

\end{document}