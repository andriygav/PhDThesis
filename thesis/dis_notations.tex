\noindent$\textbf{x}_i \in \textbf{X}$ --- вектор признакового описания $i$-го объекта\\
$y_i \in \mathbf{y}$ --- метка $i$-го объекта\\
$\mathfrak{D}$ --- выборка\\
$\textbf{X} \subset \mathbb{X}$ --- матрица, содержащая признаковое описание объектов выборки\\
$\textbf{y} \subset \mathbb{Y}$ --- вектор меток объектов выборки\\
$m$ --- количество объектов в выборке\\
$n$ --- количество признаков в признаковом описании объекта\\
$\mathbb{X} = \mathbb{R}^m$ --- признаковое пространство объектов\\
$\mathbb{Y}$ --- множество меток объектов\\
$R$ --- множество классов в задаче классификации\\
$r$ --- число оптимизаций модели\\
$(V,E)$ --- граф со множеством вершин $V$ и множеством ребер $E$\\
$\mathbf{g}^{j,k}$ --- вектор базовых функций для ребра $(j,k)$\\
$K^{j,k}$ --- мощность вектора базовых функций для ребра $(j,k)$\\
$\mathfrak{F}$ --- параметрическое семейство моделей учителя\\
$\mathfrak{G}$ --- параметрическое семейство моделей ученика\\
$\textbf{w} \in \mathbb{W}$ --- параметры модели\\
$\mathbb{W}$ --- пространство параметров модели\\
$\mathcal{L}$ --- функция потерь\\
$D_\text{KL}\bigl(p_1 || p_2\bigr)$ --- дивергенция Кульбака-Лейблера  между распределениями $p_1$ и $p_2$
$\textbf{A}^{-1}$ --- матрица ковариаций параметров модели\\