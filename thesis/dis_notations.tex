\noindent
$\textbf{x}_i \in \textbf{X}$ --- вектор признакового описания $i$-го объекта\\
$y_i \in \mathbf{y}$ --- метка $i$-го объекта\\
$\mathfrak{D}$ --- выборка для аппроксимации\\
$\mathbb{X} = \mathbb{R}^m$ --- признаковое пространство объектов\\
$\mathbb{Y}^{'}$ --- множество оценок меток\\
$\mathbb{Y}$ --- множество меток объектов\\
$\textbf{X} \subset \mathbb{X}$ --- матрица, содержащая признаковое описание объектов выборки\\
$\mathbf{I}$ --- индексное множество объектов с привилегированной информацией\\
$\textbf{y} \subset \mathbb{Y}$ --- вектор меток объектов выборки\\
$\mathbf{s} \subset \mathbb{Y}^{'}$ --- вектор оценок меток учителем\\
$m$ --- число объектов в выборке\\
$n$ --- число признаков в признаковом описании объекта\\
$R$ --- число классов в задаче классификации\\
$K$ --- число локальных моделей в ансамбле\\
$\mathfrak{F}$ --- параметрическое семейство моделей учителя\\
$\mathfrak{G}$ --- параметрическое семейство моделей ученика\\
$\mathbf{f}$ --- модель учителя\\
$\mathbf{g}$ --- модель ученика\\
$\mathbf{u}$ --- вектор параметров модели учителя\\
$\mathbf{w}$ --- вектор параметров модели ученика\\
$\textbf{w} \in \mathbb{W}$ --- параметры модели ученика\\
$\mathbb{W}$ --- пространство параметров модели ученика\\
$\textbf{u} \in \mathbb{U}$ --- параметры модели учителя\\
$\mathbb{U}$ --- пространство параметров модели учителя\\
$\mathbf{n}$ --- структура модели\\
$\mathcal{L}$ --- функция потерь\\
$p\bigr(\mathbf{u}\bigr)$ --- априорное распределение параметров учителя\\
$p\bigr(\mathbf{u}|\mathfrak{D}\bigr)$ --- апостериорное распределение параметров учителя\\
$p\bigr(\mathbf{w}\bigr)$ --- априорное распределение параметров ученика\\
$\mathbf{m}, \bm{\mu}$ --- мат. ожидание параметров учителя до и после выравнивания\\
$\bm{\Sigma}, \bm{\Xi}$ --- ковариационная матрица параметров учителя до и после выравнивания\\
$\bm{\pi}\bigr(\mathbf{x}, \mathbf{V}\bigr)$ --- шлюзовая функция в смеси экспертов\\
$\mathbf{V} \in \mathbb{V}$ --- параметры шлюзовой функции\\
$\mathbb{V}$ --- пространство параметров шлюзовой функции\\
$E$ --- экспертная информация о выборке\\
$K_{x}\bigr(\cdot, E\bigr), K_{y}\bigr(\cdot, E\bigr)$ --- отображение в признаковое описание объектов на основе экспертной информации\\