\chapter{Априорное распределения параметров моделей}

Повышение точности аппроксимации в задачах машинного обучения влечет повышение сложности моделей и снижает их интерпретируемость.
Примеры моделей с повышенной сложностью являются AlexNet~\cite{Krizhevsky2012}, VGGNet~\cite{Simonyan2014}, ResNet~\cite{Kaiming2015}, BERT~\cite{Devlin2018, Vaswani2017}, mT5~\cite{Linting2021}, GPT3~\cite{Brown2020}, а также ансамбли этих моделей.
Табл.~\ref{tb:intro:1} описывает глубокие модели машинного обучения.
Число параметров моделей машинного обучения с годами растет.
Это влечет снижение интерпретируемости моделей.
Данная проблема рассматривается в специальном классе задач по состязательным атакам (англ. adversarial attack)~\cite{Zheng2020}.

\begin{table}[h!]
\caption{Анализ роста числа параметров при развитии моделей глубокого обучения}
\label{tb:intro:1}
\begin{center}
\resizebox{\textwidth}{!}{
\begin{tabular}{|l|l|l|l|l|l|l|}
\hline
Название               & AlexNet     & VGGNet      & ResNet      & BERT     & mT5   & GPT3  \\ \hline
Год                          & 2012        & 2014        & 2015        & 2018     & 2020  & 2020  \\ \hline
Тип данных             & изображение & изображение & изображение & текст    & текст & текст \\ \hline
Число параметров, млрд & $0{,}06$    & $0{,}13$    & $0{,}06$    & $0{,}34$ & $13$  & $175$ \\ \hline
\end{tabular}
}
\end{center}
\end{table}


При построении модели машинного обучения оптимизируются два критерия: сложность модели и точность аппроксимации модели.
\begin{definition}
Сложность модели (структурная сложность)~--- число обучаемых параметров, используемых предсказтельнной моделью.
\end{definition}
Модель, которая имеет меньшую сложность при фиксированной точности, является более предпочтительной~\cite{bachteev2018}. Для снижения сложности рассматривается метод \textit{дистилляции} моделей глубокого обучения. Он строит новые модели на основе ранее обученных моделей.

\begin{definition}
Дистилляция модели --- снижение сложности модели путем выбора модели в множестве более простых моделей на основе анализа пространства параметров и предсказаний целевой переменной более сложной фиксированной модели.
\end{definition}

Исследуется проблема снижения числа обучаемых параметров моделей машинного обучения.
Большое число параметров требует больших вычислительных ресурсов.
Из-за этого данные модели не могут быть использованы в мобильных устройствах.
Идея дистилляции предложена в работах Дж.\,Е. Хинтона и В.\,Н. Вапником~\cite{Hinton2015, Vapnik2015, Lopez2016}. В этих работах предлагается использовать ответы учителя в качестве целевой переменной для обучения модели ученика.
Для снижения числа параметров предложен метод дистилляции модели~\cite{Hinton2015, Vapnik2015, Lopez2016}.

Дистиллируемая модель с большим числом параметров называется \textit{учитель}, а модель получаемая путем дистилляции называется \textit{ученик}.
При оптимизации параметров модели ученика используется модель учителя с фиксированными параметрами.

В работе~\cite{Hinton2015} Дж.\,Е. Хинтоном предлагается метод дистилляции моделей машинного обучения для задачи классификации и проведены эксперименты дистилляции моделей. Проведен эксперимент на выборке MNIST~\cite{mnist}, в котором нейросеть с избыточным числом параметров была дистиллирована в нейросеть меньшей сложности. Эксперимент по распознаванию речи, в котором ансамбль моделей дистиллирован в одну модель. Проведен эксперимент по обучению экспертных моделей на основе одной большой модели.

\begin{definition}
Привилегированная информация --- множество признаков, доступных только при выборе модели, но не в при тестировании.
\end{definition}

В работе~\cite{Vapnik2015} В.\,Н. Вапником введено понятие \textit{привилегированной информации}. В работе~\cite{Lopez2016} метод дистилляции~\cite{Hinton2015} используется вместе с привилегированным обучением~\cite{Vapnik2015}. На первом этапе обучается модель \textit{учителя} в пространстве привилегированной информации. На втором этапе обучается модель \textit{ученика} в исходном признаковом пространстве используя \textit{дистилляцию}~\cite{Hinton2015}. Для обучения строится функция ошибки специального вида, которая подробно анализируется во 2й главе. Эта функция состоит из нескольких слагаемых. Она включает ошибку учителя, ученика и регуляризирующие элементы.  Первый вариант этой функции ошибки был предложен А.\,Г. Ивахненко~\cite{Ivakhnenko1994}.

\begin{definition}
Учитель --- фиксированная модель, ответы которой используются при выборе модели ученика.
\end{definition}

\begin{definition}
Ученик --- модель, которая выбирается согласно заданного критерия качества использующего учителя.
\end{definition}



Поставлен ряд экспериментов, в которых проводилась дистилляция моделей для задачи классификации машинного обучения.
Базовый эксперимент на выборке MNIST~\cite{mnist} показал применимость метода для дистилляции избыточно сложной модели в модель меньшей сложности.
Эксперимент по дистилляции ансамбля моделей в одну модель для решения задачи распознания речи. Также в работе~\cite{Hinton2015} был проведен эксперимент по обучению экспертных моделей на основе одной модели с большим числом параметров при помощи предложенного метода дистилляции на ответах учителя.

В работе~\cite{Zehao2017} предложен метод передачи селективности нейронов (англ. neuron selectivity transfer) основаный на минимизации специальной функции потерь основаной на максимальном среднем отклонении (англ. maximum mean discrepancy) между выходами всех слоев модели учителя и ученика. Вычислительный эксперимент показал эффективность данного метода для задачи классификации изображений на примере выборок CIFAR~\cite{cifar10} и ImageNet~\cite{imagenet}.

Важным свойством дистиллированных является то, что избыточная сложность модели учителя заключается в большом числе не релевантных параметров.
\begin{definition}
Релевантность параметров --- численная характеристика описывающая влияние параметров на предсказания моделей.
\end{definition}
Предлагается удалять наименее релевантные параметры модели. Под \textit{релевантностью}~\cite{cun1990} подразумевается то, насколько параметр влияет на функцию ошибки. Малая релевантность указывает на то, что удаление этого параметра не влечет значимого изменения функции ошибки. Метод предлагает построение исходной избыточной сложности нейросети с большим количеством избыточных параметров.

В работах предлагается~\cite{cun1990,graves2011} метод введения отношения порядка на множестве параметров сложных параметрических моделей, таких как нейросеть. Рассматривается порядок, заданный при помощи ковариационной матрицы градиентов функции ошибки по параметрам модели~\cite{Mandt2017}. В работе~\cite{Chunyan2016} предложен итерационный алгоритм для поиска ковариационной матрицы градиентов. Данный итерационный алгоритм интегрируется в градиентный алгоритм оптимизации Adam~\cite{kingma2014}.

\section{Привилегированное обучение Вапника и дистилляция Хинтона}

Задано множество объектов $\bm{\Omega}$ и множество целевых переменных $\mathbb{Y}$. Множество $\mathbb{Y}=\{1,\cdots,K\}$ для задачи классификации, где $K$ число классов, множество $\mathbb{Y}=\mathbb{R}$ для задачи регрессии.
Для каждого объекта из $\omega_i \in \bm{\Omega}$ задана целевая переменная $\mathbf{y}_i = \mathbf{y}\bigr(\omega_i\bigr)$. Множество целевых переменных для всех объектов обозначим $\mathbf{Y}$.
Для множества $\bm{\Omega}$ задано отображение в некоторое признаковое пространство $\mathbb{R}^{n}$:
\[
\label{eq:st:phi}
\begin{aligned}
\varphi:\bm{\Omega} \to \mathbb{R}^{n}, \quad \left|\bm{\Omega}\right| = m,
\end{aligned}
\]
где $n$ размерность признакового пространства, а $m$ количество объектов в множестве $\bm{\Omega}$. Отображение $\varphi$ отображает объект $\omega_i \in \bm{\Omega}$ в соответствующий ему вектор признаков $\mathbf{x}_i = \varphi(\omega_i)$.
Пусть для объектов $\bm{\Omega}^* \subset \bm{\Omega}$ задана привилегированная информация:
\[
\label{eq:st:phi*}
\begin{aligned}
\varphi^*:\bm{\Omega}^* \to \mathbb{R}^{n^*}, \quad \left|\bm{\Omega}^*\right| = m^*,
\end{aligned}
\]
где $m^* \leq m$ --- число объектов с привилегированной информацией, $n^*$ --- число признаков в пространстве привилегированной информации. Отображение $\varphi^*$ отображает объект $\omega_i \in \bm{\Omega^*}$ в соответствующий ему вектор признаков $\mathbf{x}^*_i = \varphi^*(\omega_i)$.

Множество индексов объектов с известной привилегированной информацией обозначим $\mathcal{I}$:
\[
\label{eq:st:3}
\begin{aligned}
\mathcal{I} = \{1 \leq i \leq m | \text{для $i$-го объекта задана привилегированная информация}\},
\end{aligned}
\]
а множество индексов объектов с не известной привилегированной информацией обозначим $\{1, \cdots, m\}\setminus \mathcal{I} = \bar{\mathcal{I}}$.

Пусть на множестве привилегированных признаков задана функция учителя $\mathbf{f}\bigr(\mathbf{x}^*\bigr)$:
\[
\label{eq:st:4}
\begin{aligned}
\mathbf{f}:\mathbb{R}^{n^*} \to \mathbb{Y}^*,
\end{aligned}
\]
где $\mathbb{Y}^*=\mathbb{Y}$ для задачи регрессии и $\mathbb{Y}^*$ является единичным симплексом $\mathcal{S}_K$ в пространстве размерности $K$ для задачи классификации. Модель учителя $\mathbf{f}$ ставит объекты $\mathbf{X}^*$ в соответствие объектам $\mathbf{S},$ то есть  $\mathbf{f}\bigr(\mathbf{x}^*_i\bigr)=\mathbf{s}_i$.

Требуется выбрать модель ученика $\mathbf{g}\bigr(\mathbf{x}\bigr)$ из множества:
\[
\label{eq:st:G}
\begin{aligned}
\mathfrak{G} = \left\{\mathbf{g}| \mathbf{g}:\mathbb{R}^{n} \to \mathbb{Y}^*\right\},
\end{aligned}
\]
например для задачи классификации множество $\mathfrak{G}$ может быть параметрическим семейством функций линейных моделей:
\[
\label{eq:st:G:lin:cl}
\begin{aligned}
\mathfrak{G}_\text{lin,cl} = \left\{\mathbf{g}\bigr(\mathbf{W}, \mathbf{x}\bigr)| \mathbf{g}\bigr(\mathbf{W}, \mathbf{x}\bigr) = \textbf{softmax}\bigr(\mathbf{W}\mathbf{x}\bigr), \quad \mathbf{W} \in \mathbb{R}^{n\times K}\right\}.
\end{aligned}
\]

Рассмотрим описание метода предложеного в работах~\cite{Hinton2015, Lopez2016}. В рамках данных работ предполагается, что для всех данных доступна привилегированная информация $\mathcal{I} = \{1, 2, \cdots, m\}$. В работе~\cite{Hinton2015} решается задача классификации вида:
\[
    \mathfrak{D} = \{\left(\mathbf{x}_i, y_i\right)\}_{i=1}^{m}, \qquad \mathbf{x}_i \in \mathbb{R}^{n}, \quad y_i \in \mathbb{Y}=\{1, \cdots, K\},
\]
где $y_i$ --- это класс объекта, также обозначим $\mathbf{y}_i$ вектором вероятности для класса $y_i$.

В постановке Хинтона рассматривается параметрическое семейство функций:
\[
\label{eq:G:set:cl}
\mathfrak{G}_{\text{cl}} = \left\{\mathbf{g}| \mathbf{g} = \text{softmax}\bigr(\mathbf{z}\bigr(\mathbf{x}\bigr)/T\bigr), \quad \mathbf{z}: \mathbb{R}^n \to \mathbb{R}^K \right\},
\]
где $\mathbf{z}$ --- это дифференцируемая параметрическая функция заданной структуры, $T$ --- параметр температуры. В качестве модели учителя $\mathbf{f}$ рассматривается функция из множества $\mathfrak{F}_{\text{cl}}$:
\[
\label{eq:F:set:cl}
\mathfrak{F}_{\text{cl}} = \left\{\mathbf{f}| \mathbf{f} = \text{softmax}\bigr(\mathbf{v}\bigr(\mathbf{x}\bigr)/T\bigr), \quad \mathbf{v}: \mathbb{R}^n \to \mathbb{R}^K \right\},
\]
где $\mathbf{v}$ --- это дифференцируемая параметрическая функция заданной структуры, $T$ --- параметр температуры.
Параметр температуры $T$ имеет следующие свойства:
\begin{enumerate}
    \item при $T\to 0$ получаем вектор, в котором один из классов имеет единичную вероятность;
    \item при $T\to \infty$ получаем равновероятные классы.
\end{enumerate}

Функция потерь $\mathcal{L}$ учитывает перенос информации от модели учителя $\mathbf{f}$ к модели ученика $\mathbf{g}$ имеет следующий вид:
\[
\label{eq:hinton:1}
\begin{aligned}
   \mathcal{L}_{st}\bigr(\mathbf{g}\bigr) = &-\sum_{i=1}^{m}\underbrace{{\sum_{k=1}^{K}y^k_i\log\mathbf{g}\bigr(\mathbf{x}_i\bigr)\bigr|_{T=1}}}_{\text{исходная функция потерь}}\\
   &-\sum_{i=1}^{m}\underbrace{{\sum_{k=1}^{K}\mathbf{f}\bigr(\mathbf{x}_i\bigr)\bigr|_{T=T_0}\log\mathbf{g}\bigr(\mathbf{x}_i\bigr)\bigr|_{T=T_0}}}_{\text{слагаемое дистилляция}},
\end{aligned}
\]
где $\cdot\bigr|_{T=t}$ обозначает, что параметр температуры $T$ в предыдущей функции равняется $t$.

Получаем оптимизационную задачу:
\[
\label{eq:hinton:opt}
\begin{aligned}
   \hat{\mathbf{g}} = \arg\min_{\mathbf{g} \in \mathfrak{G}_{\text{cl}}} \mathcal{L}_{st}\bigr(\mathbf{g}\bigr).
\end{aligned}
\]

Работа~\cite{Lopez2016} обобщает метод предложенный в работе~\cite{Hinton2015}. Решение задачи оптимизации \eqref{eq:hinton:opt} зависит только от вектора ответов модели учителя $\mathbf{f}$. Следовательно признаковые пространства учителя и ученика могут различаться. В этом случае получаем следующую постановку задачи:
\[
    \mathfrak{D} = \left\{\left(\mathbf{x}_i, \mathbf{x}^*_i, y_i\right)\right\}_{i=1}^{m}, \qquad \mathbf{x}_i \in \mathbb{R}^{n}, \quad \mathbf{x}^*_i \in \mathbb{R}^{n^*}, \quad y_i \in \{1, \cdots, K\},
\]
где $\mathbf{x}_i$ это информация доступна на этапах обучения и контроля, а $\mathbf{x}^*_i$ это информация доступна только на этапе обучения. Модель учителя принадлежит множеству моделей $\mathfrak{F}_{cl}^*$:
\[
\label{eq:F:set:cl:priv}
\mathfrak{F}_{\text{cl}}^* = \left\{\mathbf{f}| \mathbf{f} = \text{softmax}\bigr(\mathbf{v}^*\bigr(\mathbf{x}^*\bigr)/T\bigr), \quad \mathbf{v}^*: \mathbb{R}^{n^*} \to \mathbb{R}^K \right\},
\]
где $\mathbf{v}^*$ --- это дифференцируемая параметрическая функция заданной структуры, $T$ --- параметр температуры. Множество моделей $\mathfrak{F}_{cl}^*$ отличается от множества моделей $\mathfrak{F}_{cl}$ из выражения \eqref{eq:F:set:cl}. В множестве $\mathfrak{F}_{cl}$ модели используют пространство исходных признаков, а в множестве $\mathfrak{F}_{cl}^*$ модели используют пространство привилегированных признаков. Функция потерь \eqref{eq:hinton:1} в случае модели учителя $\mathbf{f} \in \mathfrak{F}_{cl}^*$ переписывается в следующем виде:
\[
\label{eq:hinton:L:new}
\begin{aligned}
   \mathcal{L}_{st}\bigr(\mathbf{g}\bigr) = &-\sum_{i=1}^{m}{\sum_{k=1}^{K}y^k_i\log\mathbf{g}\bigr(\mathbf{x}_i\bigr)\bigr|_{T=1}}-\sum_{i=1}^{m}{\sum_{k=1}^{K}\mathbf{f}\bigr(\mathbf{x}^*_i\bigr)\bigr|_{T=T_0}\log\mathbf{g}\bigr(\mathbf{x}_i\bigr)\bigr|_{T=T_0}},
\end{aligned}
\]
где $\cdot\bigr|_{T=t}$ обозначает, что параметр температуры $T$ в предыдущей функции равняется $t$.

Требуется построить модель, которая использует привилегированную информацию $\mathbf{x}^*_i$ при обучении. Для этого рассмотрим двухэтапную модель обучения предложенную в работе~\cite{Lopez2016}:
\begin{enumerate}
    \item выбираем оптимальную модель учителя $\mathbf{f} \in \mathfrak{F}_{\text{cl}}^*$;
    \item выбираем оптимальную модель ученика $\mathbf{g} \in \mathfrak{G}_{\text{cl}}$ используя дистилляцию~\cite{Hinton2015}. 
\end{enumerate}

Модель ученика --- это функция минимизирующая \eqref{eq:hinton:L:new}. Модель учителя --- это функция минимизирующая кросс--энтропийную функции ошибки:
\[
\label{eq:hinton.2}
\begin{aligned}
   \mathcal{L}_{th}\bigr(\mathbf{f}\bigr) = &-\sum_{i=1}^{m}{{\sum_{k=1}^{K}y^k_i\log\mathbf{f}\bigr(\mathbf{x}^*_i\bigr)}}.
   \end{aligned}
\]

\section{Релевантность параметров моделей глубокого обучения}

Задана выборка
\[
\label{2.1}
\mathfrak{D} = \{\textbf{x}_i,y_i\},  i =1,...,N,
\]
где $\textbf{x}_i \in \mathbb{R}^{m}$, $y_i \in \{1, \dots, Y\}$, $Y$ --- число классов.
Рассмотрим модель $f(\mathbf{x}, \mathbf{w}): \mathbb{R}^m \times \mathbb{R}^n \to \{1,\dots,Y\}$, где $\textbf{w} \in \mathbb{R}^n$ --- пространство параметров модели,

\[
\label{2.2}
f(\mathbf{x}, \mathbf{w}) = \text{softmax}\bigl( f_1(f_2(...(f_l(\mathbf{x}, \mathbf{w})\bigr),
\]
где $f_i(\mathbf{x}, \mathbf{w}) =  \text{tanh}(\mathbf{w}^\mathsf{T}\mathbf{x})$, $l$ --- число слоев нейронной сети, $i \in \{1\dots l\}$.
Параметр $w_j$ модели $f$  называется активным, если $w_j \not = 0$. Множество индексов активных параметров обозначим $\mathcal{A} \subset \mathcal{J} = \{1,...,n\}$.
Задано пространство параметров модели:

\[
\label{2.3}
\mathbb{W_\mathcal{A}} = \{ \textbf{w} \in \mathbb{R}^n | w_j\not=0, j \in \mathcal{A}  \},
\]


Для модели $f$ с множеством индексов активных параметров $\mathcal{A}$ и соответствующего ей вектора параметров $\textbf{w} \in \mathbb{W_\mathcal{A}}$  определим логарифмическую функцию правдоподобия выборки:
\[
\label{2.4}
\mathcal{L}_\mathfrak{D}(\mathfrak{D}, \mathcal{A}, \textbf{w}) = \log p(\mathfrak{D}|\mathcal{A}, \textbf{w}),
\]
где $p(\mathfrak{D}|\mathcal{A},\textbf{w})$ --- апостериорная вероятность выборки $\mathfrak{D}$ при заданных $\textbf{w}, \mathcal{A}$.
Оптимальные значения $\textbf{w},\mathcal{A}$ находятся из минимизации $-\mathcal{L}_{\mathcal{A}}(\mathfrak{D},\mathcal{A})$ --- логарифма правдоподобия модели:
\[
\label{2.5}
\mathcal{L}_{\mathcal{A}}(\mathfrak{D},\mathcal{A}) =\log p(\mathfrak{D}|\mathcal{A}) = \log  \int_{{\textbf{w}\in\mathbb{W_\mathcal{J}}}}
p(\mathfrak{D} | \textbf{w}) p(\textbf{w} | \mathcal{A}) d \textbf{w},
\]
где $p(\textbf{w}|\mathcal{A})$ ---  априорная вероятность вектора параметров в пространстве $\mathbb{W_\mathcal{J}}$.

Так как вычисление интеграла \eqref{2.5} является вычислительно сложной задачей, рассмотрим вариационный подход~\cite{bishop2006} для решения этой задачи. Пусть задано распределение $q$:
\[
\label{2.6}
q(\textbf{w})\sim \mathcal{N}(\textbf{m}, \textbf{A}^{-1}_\text{ps}),
\]
где $\textbf{m}, \textbf{A}^{-1}_\text{ps}$ --- вектор средних и матрица ковариации, аппроксимирующее неизвестное апостериорное распределение $p(\textbf{w}|\mathfrak{D},\mathcal{A})$:
\[
\label{2.7}
p(\textbf{w} | \mathcal{A})\sim \mathcal{N}(\boldsymbol{\mu},\textbf{A}^{-1}_{\text{pr}}),
\]
где $\boldsymbol{\mu},\textbf{A}^{-1}_{\text{pr}}$ --- вектор средних и матрица ковариации.

Приблизим интеграл \eqref{2.5} методом из~\cite{bishop2006}:

\[
\label{2.8}
\begin{aligned}
\mathcal{L}_{\mathcal{A}}(\mathfrak{D},\mathcal{A}) &= \log p(\mathfrak{D}|\mathcal{A}) = \\
&=\int_{\textbf{w}\in\mathbb{W_\mathcal{J}}} q(\textbf{w}) \log \frac{p(\mathfrak{D}, \textbf{w}|\mathcal{A})}{q(\textbf{w})}d \textbf{w} - \int_{\textbf{w}\in\mathbb{W_\mathcal{J}}}  q(\textbf{w}) \log \frac{p(\textbf{w}|\mathfrak{D},\mathcal{A})}{q(\textbf{w})}d \textbf{w} \approx \\
&\approx \int_{\textbf{w}\in\mathbb{W_\mathcal{J}}} q(\textbf{w}) \log \frac{p(\mathfrak{D}, \textbf{w}|\mathcal{A})}{q(\textbf{w})}d \textbf{w} = \\
&= \int_{\textbf{w}\in\mathbb{W_\mathcal{J}}} q(\textbf{w}) \log \frac{p(\textbf{w}| \mathcal{A})}{q(\textbf{w})}d \textbf{w} + \int_{\textbf{w}\in\mathbb{W_\mathcal{J}}} q(\textbf{w}) \log p(\mathfrak{D}|\mathcal{A}, \textbf{w})d \textbf{w}=\\
&=\mathcal{L}_\textbf{w}(\mathfrak{D}, \mathcal{A}, \textbf{w})+\mathcal{L}_{E}(\mathfrak{D},\mathcal{A}).
\end{aligned}
\]

Первое слагаемое формулы \eqref{2.8} --- это сложность модели. Оно определяется расстоянием Кульбака-Лейблера:
\[
\label{2.9}
\mathcal{L}_\textbf{w}(\mathfrak{D}, \mathcal{A}, \textbf{w}) = -D_{KL}\bigl(q(\textbf{w})||p(\textbf{w}|\mathcal{A})\bigr).
\]
Второе слагаемое формулы \eqref{2.8} является матожиданием правдоподобия выборки $\mathcal{L}_\mathfrak{D}(\mathfrak{D},\mathcal{A}, \textbf{w}),$ рассматриваемое в качестве функции ошибки:
\[
\label{2.10}
\mathcal{L}_{E}(\mathfrak{D},\mathcal{A}) = \mathsf{E}_{\textbf{w}\sim q}\mathcal{L}_\mathfrak{D}(\textbf{y}, \mathfrak{D}, \mathcal{A}, \textbf{w}).
\]

Требуется найти параметры, доcтавляющие минимум суммарному функционалу потерь $\mathcal{L}_\mathcal{A}(\mathfrak{D},\mathcal{A},\textbf{w})$ из \eqref{2.8}:
\[
\label{2.11}
\begin{aligned}
\hat{\textbf{w}} &= \arg\min_{\mathcal{A}\subset\mathcal{J}, \textbf{w} \in \mathbb{W_\mathcal{A}}} -\mathcal{L}_\mathcal{A}(\mathfrak{D}, \mathcal{A}, \textbf{w}) = \\
&=\arg\min_{\mathcal{A}\subset\mathcal{J}, \textbf{w} \in \mathbb{W_\mathcal{A}}} D_{KL}\bigl(q(\textbf{w})||p(\textbf{w}|\mathcal{A})\bigr) - \mathcal{L}_\mathfrak{D}(\mathfrak{D}, \mathcal{A}, \textbf{w}).
\end{aligned}
\]

\paragraph{Случайное удаление.}
Метод случайного удаления заключается в том, что случайным образом удаляется некоторый параметр $w_\xi$ из множества активных параметров сети.  Индекс параметра $\xi$ из равномерного распределения  случайная величина, предположительно доставляющая оптимум в \eqref{2.11}.
\[
\label{3.1.1}
\xi \sim \mathcal{U}(\mathcal{A}).
\]

\paragraph{Оптимальное прореживание.}
Метод оптимального прореживания~\cite{cun1990} использует вторую производную целевой функции \eqref{2.4} по параметрам для определения нерелевантных параметров. Рассмотрим функцию потерь $\mathcal{L}$ \eqref{2.4} разложенную в ряд Тейлора в некоторой окрестности вектора параметров $\textbf{w}$:
\[
\label{3.2.1}
\delta \mathcal{L} = \sum_{j\in \mathcal{A}} g_j\delta w_j + \frac{1}{2}\sum_{i,j\in \mathcal{A}} h_{ij}\delta w_i\delta w_j + O(||\delta\textbf{w}||^3),
\]
где $\delta w_j $ --- компоненты вектора $\delta\textbf{w}$, $g_j$ --- компоненты вектора градиента $\nabla \mathcal{L}$, а $h_{ij}$ --- компоненты гесcиана $\textbf{H}$:
\[
\label{3.2.2}
g_j = \frac{\partial \mathcal{L}}{\partial w_j}, \qquad h_{ij} = \frac{\partial^2\mathcal{L}}{\partial w_i \partial w_j}.
\]

Задача является вычислительно сложной в силу размерности матрицы \textbf{H}. Введем следующее предположение~\cite{cun1990}, о том что удаление нескольких параметров приводит к такому же изменению функции потерь $\mathcal{L}$, как и суммарное изменение при индивидуальном удалении:
\[
\label{3.2.3}
\delta \mathcal{L} = \sum_{j\in \mathcal{A}} \delta \mathcal{L}_j,
\]
где $\mathcal{A}$ --- множество активных параметров, $\delta\mathcal{L}_j$ --- изменение функции потерь, при удалении одного параметра $\textbf{w}_j$.

В силу данного предположения рассматриваются только диагональные элементы матрицы \textbf{H}. После введенного предположения, выражение \eqref{3.2.1} принимает вид
\[
\label{3.2.4}
\delta \mathcal{L} = \frac{1}{2} \sum_{j\in \mathcal{A}} h_{jj}\delta w_j^2, 
\]

Получаем следующую задачу оптимизации:

\[
\label{3.2.5}
\xi = \arg\min_{j\in \mathcal{A}} h_{jj}\frac{w_j^2}{2},
\]
где $\xi$ --- индекс наименее релевантного, удаляемого параметра, предположительно доставляющая оптимум в \eqref{2.11}.

\paragraph{Удаление неинформативных параметров с помощью вариационного вывода.}
Для удаления параметров в работе~\cite{graves2011} предлагается удалить параметры, которые имеют максимальное отношение плотности $p(\textbf{w}|\mathcal{A})$ априорной вероятности в нуле к плотности вероятности априорной вероятности в математическом ожидании параметра.\\
Для гауссовского распределения с диагональной матрицей ковариации получаем:
\[
\label{3.3.1}
p_j(\textbf{w}|\mathcal{A})(x) = \frac{1}{\sqrt[]{2\sigma_j^2}}\exp({-\frac{(x-\mu_j)^2}{2\sigma_j^2}}).
\]
Разделив плотность вероятности в нуле к плотности в математическом ожидание
\[
\label{3.3.2}
\frac{p_j(\textbf{w}|\mathcal{A})(0)}{p_j(\textbf{w}|\mathcal{A})(\mu_j)}= \exp({-\frac{\mu_j^2}{2\sigma_j^2}}),
\]
Получаем следующую задачу оптимизации:
\[
\label{3.3.3}
\xi = \arg\min_{j\in \mathcal{A}} \left|\frac{\mu_j}{\sigma_j}\right|,
\]
где $\xi$ --- индекс наименее релевантного, удаляемого параметра.

\section{Смесь экспертов для аппроксимации мультимодальной выборки }

Исследуется проблема построения смеси экспертов.
Смесь экспертов является мультимоделью, состоящей из набора локальных моделей и шлюзовой функции.
Локальные модели называются~\textit{экспертами}.
Смесь экспертов использует шлюзовую функцию для взвешивания прогнозов каждого эксперта.
Весовые коэффициенты шлюзовую функции зависят от объекта, для которого производится прогноз.
Примерами мультимоделей являются бэггинг, градиентный бустинг~\cite{Tianqi2016} и случайный лес~\cite{Ishwaran2012}.
В статье~\cite{Yuksel2012} предполагается, что вклад каждого эксперта в ответ зависит от объекта из набора данных.

Большое количество работ в области построения смеси экспертов посвящены выбору шлюзовой функции: используется softmax, процесс Дирихле~\cite{Edward2002}, нейронная сеть~\cite{Shazeer2017} с функцией softmax на последнем слое. Ряд работ посвящены выбору моделей в качестве отдельных экспертов. В работах~\cite{Jordan1994, Jordan1991} в качестве модели эксперта рассматривается линейная модель.
Работы~\cite{Lima2007, Cao2003} рассматриваю модель SVM в качестве модели эксперта.
В работе~\cite{Yuksel2012} представлен обзор методов и моделей в задачах смеси экспертов.

Смеси экспертов имеют приложения в прикладных задачах. Работы~\cite{Yumlu2003, Cheung1995, Weigend2000} посвящены применению смеси экспертов в задачах прогнозирования временных рядов. 
В работе~\cite{Ebrahimpour2009} предложен метод распознавания рукописных цифр. 
Метод распознания текстов при помощи смеси экспертов иследуется в работах~\cite{Estabrooks2001}, распознание речи~\cite{Mossavat2010, Peng1996, Tuerk2001}.
В работе~\cite{Estabrooks2001} решается задача классификации текстов.
В работах~\cite{Cheung1995,Weigend2000,Cao2003,Mossavat2010,Sminchisescu2007,Tuerk2001,Yumlu2003}, используется смесь экспертов для прогнозировании временных рядов при распознавания человеческой речи, повседневной деятельности человека и прогнозирования стоимости ценных бумаг.
В работе~\cite{Ebrahimpour2009} смесь экспертов применяется для решения задачи распознавания рукописных чисел на изображениях.
В работе~\cite{Sminchisescu2007} исследуется смесь экспертов для задачи распознавания трехмерных движений человека. 
В~\cite{Bowyer2010} описаны работы по исследованию обнаружения радужки глаза на изображении. В работах~\cite{Matveev2010, Matveev2014} в частности описаны методы выделения границ радужки и зрачка.